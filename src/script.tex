\documentclass[a4paper]{article}

\usepackage{verbatim}
\input{../latex-std/lang-de.tex}



\begin{document}	
	\section{Einführung}
	
	Die niederländische Regierung entschied einst in Bezug auf Probleme mit hohem Verkehrsaufkommen, anstatt auf möglichst geringe Erwartungswerte der Fahrtzeit hinzuwirken, eher die Minimierung der Varianz der Fahrtzeit in den Fokus zu nehmen. Auf diese Weise ist man im Durchschnitt zwar länger unterwegs, erreicht aber mehr Planungssicherheit, kann also Ankunftszeiten genauer vorhersagen:
	Wenn wir mit einer gewissen Wahrscheinlichkeit $\geq p$ rechtzeitig an einem Ort ankommen möchten, dann subtrahieren wir von der gewünschten Ankunftszeit den Erwartungswert der Fahrtdauer und einen Zeitpuffer, welcher wesentlich von der Varianz abhängt.
	
	
\end{document}